\documentclass{classrep}
\usepackage[utf8]{inputenc}
\usepackage[pdftex]{graphicx}
\usepackage[polish]{babel}
\usepackage{algorithm}
\usepackage{algorithmic}
\usepackage{multicol}
\usepackage{amsmath}
\usepackage{listings}
\usepackage{array}
\usepackage{multirow}
\usepackage{url}

\studycycle{Informatyka, studia dzienne, II st.}
\coursesemester{I}

\coursename{Metody Obliczeniowe Optymalizacji}
\courseyear{2010/2011}

\courseteacher{mgr inż. Łukasz Chomątek}
\coursegroup{czwartek, 14:15}
\svnurl{xxxxxx}

\author{%
  \studentinfo{Michał Janiszewski}{169485} \and
  \studentinfo{Leszek Wach}{169513}
}

\title{Zadanie 2: Optymalizacja kierunkowa}

\floatname{algorithm}{Algorytm}

\begin{document}

\maketitle

\section{Cel zadania}
Celem zadania było napisanie programu, który dla dowolnej funkcji dwóch zmiennych znajdzie jej minimum wzdłuż wyznaczonego odcinka \ppauza dokona minimalizacji kierunkowej. Operacja ta ma odbywać się z wykorzystaniem trzech kryteriów:
\begin{enumerate}
 \item Armijo,
 \item Wolfa,
 \item Goldensteina.
\end{enumerate}

\section{Metoda rozwiązania}
W celu rozwiązania zadania stworzyliśmy skrypt programu Matlab. Skrypt prezentuje użytkownikowi okienko GUI, w którym należy wprowadzić parametry:
\begin{description}
 \item[funkcja] funkcja, która poddawana będzie optymalizacji, oznaczona jako $F$,
 \item[start] punkt początkowy, oznaczany jako $p_0$,
 \item[direction] kierunek; wektor od punktu początkowego do końcowego, oznaczany jako $d$,
 \item[c$_1$, c$_2$] parametry określające zachowanie się metod\footnote{Znaczenie tych parametrów różni się pomiędzy metodami.},
 \item[$\rho$] współczynnik zmiany kroku,
 \item[$\epsilon$] pożądana dokładność rozwiązania,
\end{description}
należy także wybrać jedno z dostępnych kryteriów.

Zadaniem programu jest poszukiwanie pewnej wartości kroku $\lambda$, która wyraża krotność wektora kierunku rozpoczynającego się w punkcie startowym. Krok ten będzie odpowiadał znalezionemu minimum. Zależność pomiędzy wartością $\lambda$, a wartością funkcji $F$ prezentuje poniższy wzór:
\begin{equation}
 f(\lambda) = F(p_0.x + \lambda \cdot d.x, p_0.y + \lambda \cdot d.y)
\end{equation}

Należy zauważyć, że wartościom kroku należącym odcinka $\overline{p_0, p_0 + d}$ odpowiadają wartości [0,~1].

Poszukiwanie wartości kroku jest metodą iteracyjną, która w każdej iteracji przybliża wartość rozwiązania, aż do osiągnięcia zadanego warunku stopu \ppauza w tym przypadku jest to kryterium stacjonarości:
\begin{equation}
 \lvert\nabla f(\lambda)\rvert \leq \epsilon
\end{equation}

Punktem $p_k$ nazywać będziemy przybliżenie rozwiązania wyznaczone w $k$-tej iteracji, zaś symbolem $\lambda_k$ oznaczać będziemy krok podjęty w $k$-tej iteracji. Zależności pomiędzy tymi zmiennymi prezentują równania:
\begin{equation}
 \lambda = \displaystyle\sum\limits_{i=0}^n \lambda_i
\end{equation}
gdzie $n$ oznacza całkowitą ilość iteracji oraz:
\begin{equation}
 p_k = p_0 + \displaystyle\sum\limits_{i=0}^k \lambda_i \cdot d
\end{equation}


\section{Kryteria}
Poza wymaganiem poprawy\footnote{Wartość $f(\lambda)$ dla każdego kolejnego oszacowania $\lambda$ powinna być mniejsza.}, stosowane są dodatkowe kryteria, które spełniają rolę kryterium stopu. Dzięki ich wykorzystaniu zapewniamy dokładniejsze i szybsze znalezienie rozwiązania.

Każde z kryteriów zaimplementowane jest jako rekurencyjna metoda, której działanie przedstawia następujący algorytm:
\begin{enumerate}
 \item Pobierz od użytkownika wszystkie dane,
 \item podstaw $\lambda_0 = 0$
 \item jak $\lambda_k$ podstaw znak pochodnej $sign(f(\lambda))$,\label{alg:start}
 \item sprawdź wybraną metodą, czy $f(\lambda + \lambda_k)$ spełnia warunki,\label{alg:one}
 \item jeśli spełnia: podstaw $\lambda_k = \rho \cdot \lambda_k$, przejdź do \ref{alg:one},
 \item jeśli nie jest spełniony warunek stacjonarości, rekurencyjnie przejdź do \ref{alg:start},
 \item zakończ obliczenia, zwróć $\lambda$ i ilość wywołań.
\end{enumerate}


\subsection{Kryterium Armijo}
Najprostszy test, kryterium Armijo, sprawdza czy spadek wartości funkcji w danej iteracji jest nie mniejszy niż jej pochodna przemnożona przez obrany parametr $c_1$. Warunek ten można zapisać w następujący sposób:
\begin{equation}
 f(\lambda + \lambda_k) \leq f(\lambda) + c_1 \cdot \lambda_k \cdot f'(\lambda)
\end{equation}

\subsection{Kryterium Wolfa}
Podstawową wadą kryterium Armijo jest ograniczanie tylko maksymalnej długości kroku, co może prowadzić do bardzo wolnej zbieżności \ppauza poprzez stosowanie możliwie najmniejszego kroku $\lambda_k$.

Kryterium Wolfa łączy kryterium Armijo z testem krzywizny, który ogranicza wartość kroku od dołu:

\begin{eqnarray}
    f(\lambda + \lambda_k)  & \leq & f(\lambda) + c_1 \cdot \lambda_k \cdot f'(\lambda) \\
    f'(\lambda + \lambda_k) & \geq & c_2 \cdot f'(\lambda)
\end{eqnarray}
gdzie $c_1 \in (0, 1)$, natomiast $c_2 \in (c_1, 1)$.

Zadaniem tego kryterium jest zagwarantowanie, że w miejscu dużego spadku funkcji, krok nie będzie mały.

\subsection{Kryterium Goldsteina}
Podobnie jak kryterium Wolfa, kryterium Goldsteina kontroluje zarówno górną jak i dolną granicę kroku zabezpieczając przed zbyt wolnym testowaniem zbieżności.

Tak jak w poprzednim przypadku, również tu wykorzystane jest kryterium Armijo do określania górnej granicy.

\begin{eqnarray}
    f(\lambda + \lambda_k) & \leq & f(\lambda) + c_1 \cdot \lambda_k \cdot f'(\lambda) \\
    f(\lambda + \lambda_k) & \geq & f(\lambda) + (1 - c_1) \cdot \lambda_k \cdot f'(\lambda)
\end{eqnarray}
gdzie $c_1 \in (0, \frac{1}{2})$.

\section{Wyniki}
Poniższe tabele prezentują wyniki otrzymane za pomocą napisanego programu. Otrzymane wyniki zostały porównane z wynikami otrzymanymi za pomocą serwisu WolframAlpha.

\subsection{Funkcja testowa 1}
Pierwsza funkcja testowa była postaci:
\begin{equation}
 F(x, y) = x^2 \cdot \sin(x) + y^2 \cdot \sin(y)
\end{equation}

Punktem startowym testów był punkt $(-4; -4)$, zaś wektor kierunku miał wartości $[2; 2]$.

% http://www.wolframalpha.com/input/?i=min%28%28x^2%29*sin%28x%29%2B%28y^2%29*sin%28y%29%29%2C+x+%3D+y%2C+-5+%3C+x+%3C+0

Znalezione przez WolframAlpha minimum na zadanym kierunku to w przybliżeniu:
\begin{equation}
 F(-2.28892, -2.28892) = -7.89060
\end{equation}


\subsubsection{Armijo}

Wyniki działania metody Armijo prezentuje tabela \ref{armijo1}.

\begin{table}
  \centering
  \caption{Wyniki działania programu dla funkcji testowej 1 dla kryterium Armijo}
  \label{armijo1}
  \begin{tabular}{|c|c|c|c|c|c|c|}
    \hline
    $c_1$ & $\rho$ & $\lambda$ & Ilość wywołań & $f(\lambda).x$ & $f(\lambda).y$ & Minimum \\
    \hline
    0.1 & 0.7 & 0.855522 & 16 & -2.28896 & -2.28896 & -7.8906 \\
    0.3 & 0.7 & 0.855532 & 6 & -2.28894 & -2.28894 & -7.8906 \\
    0.5 & 0.7 & 0.85553 & 7 & -2.28894 & -2.28894 & -7.8906 \\
    0.1 & 0.5 & 0.85553 & 7 & -2.28894 & -2.28894 & -7.8906 \\
    0.3 & 0.5 & 0.85553 & 5 & -2.28894 & -2.28894 & -7.8906 \\
    0.5 & 0.5 & 0.85553 & 7 & -2.28894 & -2.28894 & -7.8906 \\
    0.1 & 0.3 & 0.855545 & 8 & -2.28891 & -2.28891 & -7.8906 \\
    0.3 & 0.3 & 0.855543 & 7 & -2.28891 & -2.28891 & -7.8906 \\
    0.5 & 0.3 & 0.855523 & 14 & -2.28895 & -2.28895 & -7.8906 \\
    0.1 & 0.1 & 0.85553 & 18 & -2.28894 & -2.28894 & -7.8906 \\
    0.3 & 0.1 & 0.85555 & 19 & -2.2889 & -2.2889 & -7.8906 \\
    0.5 & 0.1 & 0.85553 & 26 & -2.28894 & -2.28894 & -7.8906 \\
    \hline
  \end{tabular}
\end{table}

\subsubsection{Wolf}

Wyniki działania metody Wolfa prezentuje tabela \ref{wolf1}.

\begin{table}
  \centering
  \caption{Wyniki działania programu dla funkcji testowej 1 dla kryterium Wolfa}
  \label{wolf1}
  \begin{tabular}{|c|c|c|c|c|c|c|c|}
    \hline
    $c_1$ & $c_2$ & $\rho$ & $\lambda$ & Ilość wywołań & $f(\lambda).x$ & $f(\lambda).y$ & Minimum \\
    \hline
    0.1 & 0.2 & 0.1 & 0.855538 & 5 & -2.28892 & -2.28892 & -7.8906 \\
    0.1 & 0.2 & 0.3 & 0.855526 & 5 & -2.28895 & -2.28895 & -7.8906 \\
    0.1 & 0.2 & 0.4 & 0.85553 & 5 & -2.28894 & -2.28894 & -7.8906 \\
    0.1 & 0.4 & 0.1 & 0.855536 & 8 & -2.28893 & -2.28893 & -7.8906 \\
    0.1 & 0.4 & 0.3 & 0.855522 & 8 & -2.28896 & -2.28896 & -7.8906 \\
    0.1 & 0.4 & 0.4 & 0.855541 & 7 & -2.28892 & -2.28892 & -7.8906 \\
    0.1 & 0.6 & 0.1 & 0.855541 & 10 & -2.28892 & -2.28892 & -7.8906 \\
    0.1 & 0.6 & 0.3 & 0.855537 & 9 & -2.28893 & -2.28893 & -7.8906 \\
    0.1 & 0.6 & 0.4 & 0.855541 & 7 & -2.28892 & -2.28892 & -7.8906 \\
    0.3 & 0.2 & 0.4 & 0.85553 & 5 & -2.28894 & -2.28894 & -7.8906 \\
    0.3 & 0.4 & 0.1 & 0.855536 & 8 & -2.28893 & -2.28893 & -7.8906 \\
    0.3 & 0.4 & 0.3 & 0.855522 & 8 & -2.28896 & -2.28896 & -7.8906 \\
    0.3 & 0.4 & 0.4 & 0.855527 & 7 & -2.28895 & -2.28895 & -7.8906 \\
    0.3 & 0.6 & 0.1 & 0.855541 & 10 & -2.28892 & -2.28892 & -7.8906 \\
    0.3 & 0.6 & 0.3 & 0.855536 & 8 & -2.28893 & -2.28893 & -7.8906 \\
    0.3 & 0.6 & 0.4 & 0.855539 & 8 & -2.28892 & -2.28892 & -7.8906 \\
    0.5 & 0.6 & 0.1 & 0.855541 & 10 & -2.28892 & -2.28892 & -7.8906 \\
    0.5 & 0.6 & 0.3 & 0.855536 & 8 & -2.28893 & -2.28893 & -7.8906 \\
    0.5 & 0.6 & 0.4 & 0.855539 & 8 & -2.28892 & -2.28892 & -7.8906 \\
    \hline
  \end{tabular}
\end{table}

\subsubsection{Goldstein}

Wyniki działania metody Goldsteina prezentuje tabela \ref{goldstein1}.

\begin{table}
  \centering
  \caption{Wyniki działania programu dla funkcji testowej 1 dla kryterium Goldsteina}
  \label{goldstein1}
  \begin{tabular}{|c|c|c|c|c|c|c|}
    \hline
    $c_1$ & $\rho$ & $\lambda$ & Ilość wywołań & $f(\lambda).x$ & $f(\lambda).y$ & Minimum \\
    \hline
    0.1 & 0.1 & 0.85553 & 18 & -2.28894 & -2.28894 & -7.8906 \\
    0.1 & 0.3 & 0.855545 & 8 & -2.28891 & -2.28891 & -7.8906 \\
    0.1 & 0.5 & 0.85553 & 7 & -2.28894 & -2.28894 & -7.8906 \\
    0.1 & 0.7 & 0.855522 & 16 & -2.28896 & -2.28896 & -7.8906 \\
    0.2 & 0.1 & 0.85553 & 8 & -2.28894 & -2.28894 & -7.8906 \\
    0.2 & 0.3 & 0.855536 & 8 & -2.28893 & -2.28893 & -7.8906 \\
    0.2 & 0.5 & 0.85553 & 5 & -2.28894 & -2.28894 & -7.8906 \\
    0.2 & 0.7 & 0.855547 & 9 & -2.28891 & -2.28891 & -7.8906 \\
    0.3 & 0.1 & 0.855534 & 7 & -2.28893 & -2.28893 & -7.8906 \\
    0.3 & 0.3 & 0.855538 & 7 & -2.28892 & -2.28892 & -7.8906 \\
    0.3 & 0.5 & 0.85553 & 5 & -2.28894 & -2.28894 & -7.8906 \\
    0.3 & 0.7 & 0.855532 & 6 & -2.28894 & -2.28894 & -7.8906 \\
    \hline
  \end{tabular}
\end{table}


\subsection{Funkcja testowa 2}
Pierwsza funkcja testowa była postaci:
\begin{equation}
 F(x, y) = x^2 \cdot y^2 + \cos(y) + \sin(x)
\end{equation}

Punktem startowym testów był punkt $(-2; -6)$, zaś wektor kierunku miał wartości $[2; 6]$.

% http://www.wolframalpha.com/input/?i=min%28%28x^2%29*%28y^2%29+%2B+cos%28y%29+%2B+sin%28x%29%29%2C+3x+%3D+y

Znalezione przez WolframAlpha minima na tym kierunku to w przybliżeniu:
\begin{eqnarray}
 F(-0.47496, -1.42490) & = & 0.14610 \\ 
 F( 0.36317,  1.08952) & = & 0.97471
\end{eqnarray}


\subsubsection{Armijo}

Wyniki działania metody Armijo prezentuje tabela \ref{armijo2}.

\begin{table}
  \centering
  \caption{Wyniki działania programu dla funkcji testowej 2 dla kryterium Armijo}
  \label{armijo2}
  \begin{tabular}{|c|c|c|c|c|c|c|}
    \hline
    $c_1$ & $\rho$ & $\lambda$ & Ilość wywołań & $f(\lambda).x$ & $f(\lambda).y$ & Minimum \\
    \hline
    0.1 & 0.1 & 0.76251 & 15 & -0.47498 & -1.42494 & 0.146102 \\
    0.1 & 0.3 & 0.762509 & 7 & -0.474981 & -1.42494 & 0.146102 \\
    0.1 & 0.5 & 0.762512 & 7 & -0.474976 & -1.42493 & 0.146102 \\
    0.2 & 0.1 & 0.76251 & 15 & -0.47498 & -1.42494 & 0.146102 \\
    0.2 & 0.3 & 0.762509 & 7 & -0.474981 & -1.42494 & 0.146102 \\
    0.2 & 0.5 & 0.762512 & 7 & -0.474976 & -1.42493 & 0.146102 \\
    0.3 & 0.1 & 0.76251 & 21 & -0.47498 & -1.42494 & 0.146102 \\
    0.3 & 0.3 & 0.762521 & 11 & -0.474957 & -1.42487 & 0.146102 \\
    0.3 & 0.5 & 0.762512 & 7 & -0.474976 & -1.42493 & 0.146102 \\
    0.4 & 0.1 & 0.76251 & 21 & -0.47498 & -1.42494 & 0.146102 \\
    0.4 & 0.3 & 0.762525 & 11 & -0.47495 & -1.42485 & 0.146102 \\
    0.4 & 0.5 & 0.762512 & 7 & -0.474976 & -1.42493 & 0.146102 \\
    \hline
  \end{tabular}
\end{table}

\subsubsection{Wolf}

Wyniki działania metody Wolfa prezentuje tabela \ref{wolf2}.

\begin{table}
  \centering
  \caption{Wyniki działania programu dla funkcji testowej 2 dla kryterium Wolfa}
  \label{wolf2}
  \begin{tabular}{|c|c|c|c|c|c|c|c|}
    \hline
    $c_1$ & $c_2$ & $\rho$ & $\lambda$ & Ilość wywołań & $f(\lambda).x$ & $f(\lambda).y$ & Minimum \\
    \hline
    0.1 & 0.2 & 0.1 & 0.76251 & 15 & -0.47498 & -1.42494 & 0.146102 \\
    0.1 & 0.2 & 0.3 & 0.762509 & 7 & -0.474981 & -1.42494 & 0.146102 \\
    0.1 & 0.2 & 0.5 & 0.762512 & 7 & -0.474976 & -1.42493 & 0.146102 \\
    0.1 & 0.4 & 0.1 & 0.76251 & 15 & -0.47498 & -1.42494 & 0.146102 \\
    0.1 & 0.4 & 0.3 & 0.762509 & 7 & -0.474981 & -1.42494 & 0.146102 \\
    0.1 & 0.4 & 0.5 & 0.762512 & 7 & -0.474976 & -1.42493 & 0.146102 \\
    0.1 & 0.6 & 0.1 & 0.76251 & 15 & -0.47498 & -1.42494 & 0.146102 \\
    0.1 & 0.6 & 0.3 & 0.762509 & 7 & -0.474981 & -1.42494 & 0.146102 \\
    0.1 & 0.6 & 0.5 & 0.762512 & 7 & -0.474976 & -1.42493 & 0.146102 \\
    0.2 & 0.4 & 0.1 & 0.76251 & 15 & -0.47498 & -1.42494 & 0.146102 \\
    0.2 & 0.4 & 0.3 & 0.762509 & 7 & -0.474981 & -1.42494 & 0.146102 \\
    0.2 & 0.4 & 0.5 & 0.762512 & 7 & -0.474976 & -1.42493 & 0.146102 \\
    0.2 & 0.6 & 0.1 & 0.76251 & 15 & -0.47498 & -1.42494 & 0.146102 \\
    0.2 & 0.6 & 0.3 & 0.762509 & 7 & -0.474981 & -1.42494 & 0.146102 \\
    0.2 & 0.6 & 0.5 & 0.762512 & 7 & -0.474976 & -1.42493 & 0.146102 \\
    0.3 & 0.4 & 0.1 & 0.76251 & 21 & -0.47498 & -1.42494 & 0.146102 \\
    0.3 & 0.4 & 0.3 & 0.762521 & 11 & -0.474957 & -1.42487 & 0.146102 \\
    0.3 & 0.4 & 0.5 & 0.762512 & 7 & -0.474976 & -1.42493 & 0.146102 \\
    0.3 & 0.6 & 0.1 & 0.76251 & 21 & -0.47498 & -1.42494 & 0.146102 \\
    0.3 & 0.6 & 0.3 & 0.762521 & 11 & -0.474957 & -1.42487 & 0.146102 \\
    0.3 & 0.6 & 0.5 & 0.762512 & 7 & -0.474976 & -1.42493 & 0.146102 \\
    0.4 & 0.6 & 0.1 & 0.76251 & 21 & -0.47498 & -1.42494 & 0.146102 \\
    0.4 & 0.6 & 0.3 & 0.762525 & 11 & -0.47495 & -1.42485 & 0.146102 \\
    0.4 & 0.6 & 0.5 & 0.762512 & 7 & -0.474976 & -1.42493 & 0.146102 \\
    \hline
  \end{tabular}
\end{table}

\subsubsection{Goldstein}

Wyniki działania metody Goldsteina prezentuje tabela \ref{goldstein2}.

\begin{table}
  \centering
  \caption{Wyniki działania programu dla funkcji testowej 2 dla kryterium Goldsteina}
  \label{goldstein2}
  \begin{tabular}{|c|c|c|c|c|c|c|}
    \hline
    $c_1$ & $\rho$ & $\lambda$ & Ilość wywołań & $f(\lambda).x$ & $f(\lambda).y$ & Minimum \\
    \hline
    0.1 & 0.1 & 0.762522 & 7 & -0.474957 & -1.42487 & 0.146102 \\
    0.1 & 0.3 & 0.76252 & 7 & -0.47496 & -1.42488 & 0.146102 \\
    0.1 & 0.5 & 0.762516 & 12 & -0.474968 & -1.4249 & 0.146102 \\
    0.2 & 0.1 & 0.762521 & 6 & -0.474959 & -1.42488 & 0.146102 \\
    0.2 & 0.3 & 0.76252 & 7 & -0.47496 & -1.42488 & 0.146102 \\
    0.2 & 0.5 & 0.762515 & 8 & -0.474969 & -1.42491 & 0.146102 \\
    0.3 & 0.1 & 0.76252 & 7 & -0.47496 & -1.42488 & 0.146102 \\
    0.3 & 0.3 & 0.762522 & 9 & -0.474955 & -1.42487 & 0.146102 \\
    0.3 & 0.5 & 0.762512 & 7 & -0.474976 & -1.42493 & 0.146102 \\
    0.4 & 0.1 & 0.762516 & 6 & -0.474967 & -1.4249 & 0.146102 \\
    0.4 & 0.3 & 0.762524 & 6 & -0.474952 & -1.42485 & 0.146102 \\
    0.4 & 0.5 & 0.762512 & 5 & -0.474976 & -1.42493 & 0.146102 \\
    \hline
  \end{tabular}
\end{table}

\section{Wnioski}
Podczas testowania stworzonego programu napotkaliśmy duże problemy z takim dobraniem parametrów optymalizacji, aby rozwiązanie zostało w ogóle odnalezione. Duża część testów została odrzucona ze względu na brak wyniku.

Powodem takiego stanu rzeczy może być takie dobranie dostępnych opcji (w tym optymalizowanej funkcji), że krok wywołania jest bardzo mały, a przez to przekroczony jest limit rekurencji.

Innym powodem powyższego błędu może być przyjęta metoda różniczkowania \ppauza dokonujemy obliczenia wartości funkcji w trzech punktach i wykorzystując je szacujemy wartość pochodnej. W niektórych przypadkach metoda ta może dawać niepożądane rezultaty.

\smallskip

Dla wszystkich zestawów testowych ilość wymaganych do wykonania kroków jest dość mała, w związku z czym trudno jest jednoznacznie określić istnienie tendencji.

Wydaje się jednak, że najlepsze rezultaty uzyskiwane są dla parametru $\rho$ równego 0,5 dla wszystkich metod.

\smallskip

W przypadku kryterium Armijo efekt zmiany $c_1$ zależny jest od parametru $\rho$.

Zmiana tego parametru przy $\rho$ w okolicy 0.5 nie przynosi widocznych zmian, jednak gdy $\rho$ jest małe, zmniejszanie $c_1$ zmniejsza też wymaganą ilość kroków.

\smallskip

Kryterium Wolfa jest znacznie bardziej podatne na zmianę parametru $c_2$ \ppauza wraz ze wzrostem tej wartości wzrasta konieczna ilość kroków do rozwiązania. Optymalne do dobrania parametry to małe $c_1$, np. 0,1 oraz $\rho$ w okolicach 0,5.

\smallskip

Kryterium Goldsteina daje najlepsze rezultaty \ppauza dochodzi do rozwiązania w średnio najmniejszej ilości kroków. W przypadku tej metody widoczna jest poprawa w ilości kroków wraz z dążeniem parametrów $c_1$ oraz $\rho$ do 0,5.

\begin{thebibliography}{99}
\bibitem{grega.wyklad}
Grega, Wojciech. \textit{Metody optymalizacji. Wykład 4} [online]. [dostęp: 27
marca 2011]. Dostępny w Internecie:
http://aq.ia.agh.edu.pl/Aquarium/Dydaktyk/Wyklady/MO/2005-06/Wyklad04.PDF
\end{thebibliography}

\end{document}
