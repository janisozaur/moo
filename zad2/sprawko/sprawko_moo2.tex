\documentclass{classrep}
\usepackage[utf8]{inputenc}
\usepackage[pdftex]{graphicx}
\usepackage[polish]{babel}
\usepackage{algorithm}
\usepackage{algorithmic}
\usepackage{multicol}
\usepackage{amsmath}
\usepackage{listings}
\usepackage{array}
\usepackage{multirow}
\usepackage{url}

\studycycle{Informatyka, studia dzienne, II st.}
\coursesemester{I}

\coursename{Metody Obliczeniowe Optymalizacji}
\courseyear{2010/2011}

\courseteacher{xxxxxx}
\coursegroup{czwartek, 14:15}
\svnurl{xxxxxx}

\author{%
  \studentinfo{Michał Janiszewski}{169485} \and
  \studentinfo{Leszek Wach}{xxxxxx}
}

\title{Zadanie 2: Optymalizacja kierunkowa}

\floatname{algorithm}{Algorytm}

\begin{document}

\maketitle

\section{Cel zadania}
Celem zadania było napisanie programu, który dla dowolnej funkcji dwóch zmiennych znajdzie jej minimum wzdłuż wyznaczonego odcinka \ppauza dokona minimalizacji kierunkowej. Operacja ta ma odbywać się z wykorzystaniem trzech kryteriów:
\begin{enumerate}
 \item Armijo,
 \item Wolfa,
 \item Goldensteina.
\end{enumerate}

\section{Metoda rozwiązania}
W celu rozwiązania zadania stworzyliśmy skrypt programu Matlab. Skrypt prezentuje użytkownikowi okienko GUI, w którym należy wprowadzić parametry:
\begin{description}
 \item[funkcja] funkcja, która poddawana będzie optymalizacji,
 \item[start] punkt początkowy, oznaczany jako $x_0$
 \item[direction] kierunek; wektor od punktu początkowego do końcowego, oznaczany jako $\overrightarrow{d}$
 \item[c$_1$, c$_2$] parametry określające zachowanie się metod\footnote{Znaczenie tych parametrów zostanie omówione wraz z metodami.},
 \item[$\rho$] współczynnik zmiany kroku,
 \item[$\epsilon$] pożądana dokładność rozwiązania,
\end{description}
należy także wybrać jedno z dostępnych kryteriów.

\section{Kryteria}
Zadaniem programu jest poszukiwanie pewnej wartości kroku $\lambda$, która wyraża krotność wektora kierunku rozpoczynającego się w punkcie startowym. Krok ten będzie odpowiadał znalezionemu minimum.

Poszukiwanie wartości kroku jest metodą iteracyjną, która w każdej iteracji przybliża wartość kroku, aż do osiągnięcia zadanego warunku stopu \ppauza w tym przypadku jest to kryterium stacjonarości:
\begin{equation}
 \lvert\nabla F(x_0 + \lambda * \overrightarrow{d})\rvert \leq \epsilon
\end{equation}


\begin{thebibliography}{99}
\bibitem{grega.wyklad}
Grega, Wojciech. \textit{Metody optymalizacji. Wykład 4} [online]. [dostęp: 27
marca 2011]. Dostępny w Internecie:
http://aq.ia.agh.edu.pl/Aquarium/Dydaktyk/Wyklady/MO/2005-06/Wyklad04.PDF
\end{thebibliography}

\end{document}
